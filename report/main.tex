\documentclass[a4paper, 10pt]{article}
\usepackage[english,brazil]{babel}
\usepackage[utf8]{inputenc}
\usepackage{fullpage}

\usepackage{multicol}
\setlength{\columnsep}{30pt}


% Definitions an Theorems: ------------
\usepackage{amsthm}

\theoremstyle{plain}
\newtheorem{statement}{Enunciado}

\theoremstyle{definition}
\newtheorem{discussion}{Discussão}

\theoremstyle{definition}
\newtheorem{answer}{Resposta}
% ----------------------------------------------

% Definitions of languages: ------------
\usepackage{listings}
\lstdefinestyle{cStyle}{
  basicstyle=\scriptsize,
  breakatwhitespace=false,
  breaklines=true,
  captionpos=b,
  keepspaces=true,
  numbers=left,
  numbersep=5pt,
  showspaces=false,
  showstringspaces=false,
  showtabs=false,
  tabsize=2
}
\renewcommand*{\lstlistingname}{Código}

% ----------------------------------------------


\title{Laboratório 3 \\ Otimização com Métodos de Busca Local}
\author{CT-213 \\ Prof. Marcos Ricardo Omena de Albuquerque Maximo  \\ \\ \\  Carlos Matheus Barros da Silva \\carlosmatheusbs@gmail.com \\ \\ \\ \\ \\ \\ \\ \\ \\ \\ \\ \\ \\ \\ \\ \\ \\ \\ \\ \\ \\ \\ \\ \\ \\ \\ \\ \\}
\date{Março - 2019}


\begin{document}

\Huge
\maketitle
\newpage
\normalsize

\begin{multicols}{2}

%-------------------------------------------------------------------------------

\section*{Introdução}

Foram estudados e implementados três métodos de otimização que possuem algoritmos baseados em busca local. Esses métodos foram Gradient Descent, Hill Climbing e Simulated Anneling.

Os métodos foram testados em um problema proposto no roteiro dessa atividade[]. Esse problema utilizou regressão linear para ober parâmetros físicos relativos ao movimento de uma bola.

Como o problema tratado possui solução analítica, o experimento é a titulo educacional, já que, dessa forma, pode-se obter facilmente sua solução pelo Método dos Mínimos Quadrados (MMQ).

De maneira geral, o Laboratório foi desenvolvido com sucesso. Em cada uma das sessões seguintes encontram-se o respectivo desenvolvimento de um dos métodos estudado.

\section*{Gradient Descent}

\begin{statement}
Faca um programa em que três processos executam paralelamente as seguintes ações:
\begin{itemize}
	\item Pai - Imprime os números de 1 a 50, com um intervalo de 2 segundos entre cada numero. Apos imprimir todos os números, imprime a frase "Processo pai vai morrer".
	\item Filho1 - Imprime os números de 100 a 200, com um intervalo de 1 segundo entre cada numero. Antes de imprimir os números, imprime a frase "Filho 1 foi criado", e após imprimir todos os números, imprime a frase "Filho 1 vai morrer".
	\item Neto1- (filho do processo Filho1) imprime os números de 300 a 350, com um intervalo de 2 segundos entre cada numero. Antes de imprimir os números, imprime a frase "Neto 1 foi criado" Apos imprimir todos os números, imprime a frase "Neto 1 vai morrer".
\end{itemize}

Em cada printf os processos devem imprimir o seu pid e o pid do seu pai.

Mostre as saídas enfatizando os processos executando em "paralelo".
\end{statement}

\begin{answer}
O Problema 1 foi resolvido com sucesso. Seu código pode ser conferido no Código \ref{codeproblem1}. Seu funcionamento pode ser observado pela saída representada pelo Código \ref{outputproblem1}. As respostas às perguntas referentes a esse problema podem ser vistas na Discussão \ref{dicussionproblem1}.
\end{answer}

%\lstinputlisting[language=C, caption={Código do Problema 1}, label={codeproblem1}, style=cStyle]{programa1.c}
%\lstinputlisting[language=C, caption={Saída do Problema 1}, label={outputproblem1}, style=cStyle]{programa1.output}
%
\begin{discussion}
\label{dicussionproblem1}
O resultado foi obtido dessa forma devido ao fato de que a uma das primeiras coisas que o programa pai faz é chamar \textit{fork()} de modo que logo em seguida seu filho chama \textit{fork()} também. Então ficaram os 3 processos rodando paralelamente, de modo que cada um desses 3 começaram a fazer o que estava descrito no enunciado do problema.

Dessa forma, como pôde ser visto na saída do problema, os três processos estavam executando paralelamente. Como cada um desses 3 processos \textit{printava} ``simultaneamente'' foi visto que o Neto foi \textit{printado} ante do filho, o que é algo possível, já que como todos estão rodando mais ou menos na mesma posição o que chamar a função \textit{printf} primeiro seria \textit{printado} antes.

Os três processos ficaram em seus ciclos \textit{printando} de acordo com o previsto, até que eles morreram. O pai, como era previsto, morreu primeiro, já que ele havia começado a \textit{printar} antes. Logo em seguida, o filho e o neto também terminaram.
\end{discussion}

%-------------------------------------------------------------------------------------------------------

\section*{Problema  2}

\begin{statement}
Altere o programa 1 para que primeiro sejam exibidos primeiramente apenas os prints do neto, depois so os do filho e por ultimo so os do pai.
\end{statement}

\begin{answer}
O Problema 2 foi resolvido com sucesso. Seu código pode ser conferido no Código \ref{codeproblem2}. Seu funcionamento pode ser observado pela saída representada pelo Código \ref{outputproblem2}. As respostas às perguntas referentes a esse problema podem ser vistas na Discussão \ref{dicussionproblem2}.
\end{answer}

%\lstinputlisting[language=C, caption={Código do Problema 2}, label={codeproblem2}, style=cStyle]{programa2.c}
%\lstinputlisting[language=C, caption={Saída do Problema 2}, label={outputproblem2}, style=cStyle]{programa2.output}

\begin{discussion}
\label{dicussionproblem2}
A alteração necessária para a mudança requisitada foi bem simples e direta, já que o que foi pedido foi que rodasse o neto antes do filho, bastava fazer o filho esperar terminar a execução do neto para então executar o que ele ia fazer, e como foi pedido também que o filho rodasse antes do pai, bastava que o pai, também, esperasse o termino da execução do filho para que então executasse o que ia executar.

Portanto foi necessário apenas acrescentar duas linhas com a função \textit{wait()}, a Linha $34$ e a Linha $40$.
\end{discussion}

%-------------------------------------------------------------------------------------------------------

\end{multicols}

\end{document}
